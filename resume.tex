\documentclass[10pt,a4paper,ragged2e,withhyper]{altacv}
%% and packages.
%% See http://texdoc.net/pkg/fontawesome5 and http://texdoc.net/pkg/academicons for full list of symbols. You MUST compile with XeLaTeX or LuaLaTeX if you want to use academicons.

% Change the page layout if you need to
\geometry{left=1cm,right=1cm,top=0.5cm,bottom=0.5cm}

% The paracol package lets you typeset columns of text in parallel
\usepackage{paracol}


% Change the font if you want to, depending on whether
% you're using pdflatex or xelatex/lualatex
\ifxetexorluatex
  % If using xelatex or lualatex:
  \setmainfont{Lato}
\else
  % If using pdflatex:
  \usepackage[default]{lato}
\fi

% \usepackage[OT1]{fontenc}
% \renewcommand*\familydefault{\sfdefault} %% Only if the base font of the document is to be sans serif
% \usepackage{cmbright}
% \usepackage[OT1]{fontenc}


% prevent hyphenation (bad for .txt conversion)
\usepackage{hypernat}
\hyphenpenalty 10000
\exhyphenpenalty 10000


% Change the colours if you want to
% \colorlet{name}{black}
% \definecolor{MintyGreen}{HTML}{19b879} %  {33b884} 
\colorlet{MintyGreen}{cyan!70!black}
\colorlet{AccentPurple}{cyan!50!purple}

\colorlet{accent}{AccentPurple}
\colorlet{headingrule}{AccentPurple}
\colorlet{tagline}{AccentPurple}
\colorlet{link}{AccentPurple!70!blue}

\definecolor{LightGrey}{HTML}{000000}
\colorlet{subtle}{LightGrey}
\colorlet{body}{LightGrey}

% \definecolor{SlateGrey}{HTML}{2E2E2E}
% \colorlet{heading}{MintyGreen}
% \colorlet{subheading}{PastelRed}
% \colorlet{emphasis}{SlateGrey}
% \colorlet{body}{LightGrey}


% make pdftotxt output nicer
\input{glyphtounicode}
\pdfgentounicode=1



% Change some fonts, if necessary
% \renewcommand{\namefont}{\Huge\rmfamily\bfseries}
% \renewcommand{\personalinfofont}{\footnotesize}
% \renewcommand{\cvsectionfont}{\LARGE\rmfamily\bfseries}
% \renewcommand{\cvsubsectionfont}{\large\bfseries}

% Change the bullets for itemize and rating marker
% for \cvskill if you want to
\renewcommand{\itemmarker}{{\small\textbullet}}
\renewcommand{\ratingmarker}{\faCircle}

%% sample.bib contains your publications

\hbadness = 10001
\vbadness = 10001
\hfuzz=\maxdimen \tolerance=10000

% disable ligatures for better ats
\usepackage{microtype}
\DisableLigatures[f]{encoding = *, family = *} 

% \usepackage{setspace}
% \setstretch{1.05}

\begin{document}
\name{Brian Chen}
% \tagline{~~|~~Seeking Summer 2023 Internships}
\tagline{}
%% You can add multiple photos on the left or right
% \photoL{2cm}{Yacht_High,Suitcase_High}
\personalinfo{%
  \email{example@example.com}
  \homepage{chenbrian.ca}
  \github{github.com/ihasdapie} 
  \linkedin{linkedin.com/in/brianchen28914}
  \phone{+x xxx-xxx-xxxx}
  %% \printinfo{symbol}{detail}[optional hyperlink prefix]
}



\makecvheader


%% Depending on your tastes, you may want to make fonts of itemize environments slightly smaller
\AtBeginEnvironment{itemize}{\small}

\raggedright

% \cvsection{Summary}
% Third year Engineering Science student in Computer Engineering major at the University of Toronto with a demonstrated passion for software engineering, who is a big Linux \& open source aficionado and avid badminton player.
% I have worked in development and leadership positions in industry, research, and my own non-profit startup. 
% Interested in working in a community to identify and resolve problems in the world around us.

\cvsection{Education}{}
\cveventedu{Computer Engineering}{University of Toronto}{2020 -- 2024}{Toronto, ON}
\begin{itemize}
  \item Third year B.A.S.c in Engineering Science; Computer Engineering major, Robotics minor. Dean's list, 3.65 GPA
  \item \textbf{Coursework:} 
    \footnotesize{
     Computer Security,                                % ECE568
     Operating Systems,                                % ECE353
     Advanced Algorithm Design, % Advanced Algorithm Design  % CSC473,
     Machine Learning,                                 % ECE421
     Software Engineering,                             % ECE444
     Foundations of Computing,                         % ECE358
     Computer organization,                            % ECE352
     Control Theory,                                   % ECE356
     Semiconductor Devices,                            % ECE350
     Signal Analysis,              % ECE355
     Electronics,                                       % ECE360
     % Communication Systems,                            % ECE363
     Economics, Engineering \& Law
  } 
\end{itemize}
% \cvevent{}{Langara College}{}{}
% Concurrent enrolment with studies at Eric Hamber Secondary
% \cvevent{}{Eric Hamber Secondary}{2015 -- 2020}{Vancouver, BC}



\cvsection{Experience}{}

% (S)ituation (T)ask (A)ction (R)esult
\cvevent{Software Engineering Intern}{Open Source Robotics Foundation}{May 2022 -- September 2022}{Mountain View, CA}{\hspace{1cm}[ \textit{c, c++, python, ROS2, open source, Linux, distributed systems} ]}
\begin{itemize}
  \item Developed \textbf{60+ features and bugfixes} in collaboration with \textbf{NASA} while balancing open source community feedback for the \chref{https://docs.ros.org/en/rolling/}{ROS2} and \chref{https://gazebosim.org/home}{Gazebo} packages powering the \textbf{\chref{https://www.nasa.gov/viper}{VIPER} lunar rover's} critical ground control and autonomy systems -- \textit{leaving earth in 2024!}
\item \textbf{Co-authored \chref{https://github.com/ros-infrastructure/rep/pull/360}{REP2012: Service Introspection} standard} proposing new core functionality for \textbf{runtime introspection and recording of ROS2 services}. Designed, built, and deployed reference implementation with few iterations to \textbf{unblock tens of thousands of users while balancing stakeholder priorities} from the open source community, Open Robotics, and the Technical Steering Committee.
    % This \textbf{widely-requested feature} garnered strong community support because it 
  \item Onboarded quickly on ROS2 \& Gazebo codebases; improved the development experience for \textbf{800,000+ users} by \textbf{fixing race conditions} in ROS2, starting a mypy compliance initiative, adding an AsyncParameterClient interface, and \textbf{improving test coverage}
  

    % 10th most commented PR on the repo of all time lmao
    % well received by community
  % \item Streamlined editor integration and user experience by spearheading initiative to make rclpy \texttt{mypy} compliant
  %   Resolved race conditions in core rclpy client library,
  %   implemented \texttt{rclpy} AsyncParameterClient interface
  %   writing tut, docs, etc/
  % \item Core ROS2 \& Gazebo maintenance tasks; reducing message spam, resolving deadlocks in specific use cases (rclpy shutdown), rosbag improvements, typing fixes, empy templating syntax
  % \item Tutorials, docs, changed docs theme
\end{itemize}

{\color{accent}\hrule \vspace{0.1cm}}
\cvevent{Autonomy Software Lead}{aUToronto (University of Toronto Self-Driving Car Team)}{September 2020 -- June 2023}{Toronto, ON}{[ \textit{c++, python, Simulink, PyTorch, TensorRT, CI/CD, ROS2} ]}
\begin{itemize}
  \item \textbf{Leading 20+ students} across trajectory motion planning, automated simulation testing, and deep learning acceleration teams to build a Level 4 autonomous vehicle for \chref{https://www.autodrive.utoronto.ca/}{aUToronto}'s entry to the \chref{https://www.sae.org/attend/student-events/autodrive-challenge-series2/}{AutoDrive Challenge}. \textbf{Our team has won 1st place for the past 5 years}
  \item \textbf{Worked to a leadership position} through proactive self-learning, mentoring new members, and taking on ownership of projects
  \item Designing and implementing \textbf{time-critical trajectory motion planning} solution for our vehicle. \textbf{Led solution convergence process} to pick design through decision matrices weighing literature review, compute restrictions, failure modes, and team buy-in
    % rrt, a* waypoint, centerline, function opt
    % Designed \textbf{time-critical} vehicle \textbf{trajectory motion planning} algorithms to generate kinematically feasible trajectories using hybrid A*
  \item Accelerated YOLOv5 by 20x with a TensorRT ML pipeline to \textbf{detect objects in real-time} across 4 cameras with \textbf{millisecond latency}
  \item \textbf{Reduced developer testing time by 10x} by developing "aUToTest", a parallelized automated simulation integration test framework.
  \item Built AI sensor noise modelling tool on CycleGAN to improve Sim2Real transfer, build test confidence, and \textbf{deliver simulation value}
\end{itemize}
% \vspace{-0.3cm}
    % \item Presented work at 2021 Vector Institute Mobility Symposium \& 2021 UofT Robotics Institute AV workshop

{\color{accent}\hrule \vspace{0.1cm}}
\cvevent{Fullstack Software Developer}{BC Parks Foundation}{July 2020 -- September 2021}{Vancouver, BC}{[ \textit{python (Django), PostgreSQL, Vue.js, fullstack, GIS} ]}
\begin{itemize}
  \item \textbf{Translated multiple stakeholder needs into functional requirements and practical tasks} to build fullstack `DiscoverParks' webapp and data collection solution. I was responsible for the internal content management interface, backend, and front-end experiences
  \item Applied profiling to rearchitect database to better model user data and remove cycles; \textbf{improved code health and query speed}
\end{itemize}

{\color{accent}\hrule \vspace{0.1cm}}
\cvevent{Teaching Assistant}{Division of Engineering Science - University of Toronto}{September 2021 - June 2022} {Toronto, ON}{[ \textit{c, python} ]}
\begin{itemize}
  \item \textbf{Taught \textasciitilde50 undergrads computer science} from `Hello World' to dynamic programming and Dijkstra's algorithm (\chref{https://www.cs.toronto.edu/~guerzhoy/180/}{ESC180}, \chref{https://engineering.calendar.utoronto.ca/course/esc190h1}{ESC190})
\end{itemize}
% \vspace{-0.05cm}

{\color{accent}\hrule \vspace{0.1cm}}
\cvevent{Co-Founder \& Developer}{GrocerCheck Foundation}{April 2020 -- December 2020}{Vancouver, BC}{[ \textit{python (Django), PostgreSQL, aws} ]}
% TODO(ihasdapie): Add mapping keyword and geographic
\begin{itemize}
  \item \textbf{Created \chref{https://grocercheck.ca}{grocercheck.ca}}, a webapp that leverages big data to \textbf{help 20,000+ daily users \#ShopSafeStaySafe} by finding the least busy place to shop for groceries in \textbf{15,000+} stores across North America in response to the COVID-19 pandemic
  \item Founded GrocerCheck Foundation, a \textbf{registered non-profit} to better scale project; secured support valued at \textbf{\$200,000+}
  \item \textbf{Architected and deployed horizontally scalable distributed system architecture} to meet unexpected growth and demand
\end{itemize}

{\color{accent}\hrule \vspace{0.1cm}}
\cvevent{Research Intern}{Intelligent Sensory Microsystems Lab - University of Toronto} {Feb 2021 -- September 2021} {Toronto, ON}{\hspace{-1cm}[ \textit{python (PyTorch), neuromorphic computers, supercomputers} ]}
\begin{itemize}
  \item Innovated novel `thresholding' concept which \textbf{improves longevity and power consumption} characteristics of neuromorphic \chref{https://en.wikipedia.org/wiki/Memristor}{memristor} crossbar \textbf{machine learning accelerators} during in-situ training by \textbf{up to 90\%}. \textbf{First author paper} under review
% \item First author paper on novel `thresholding' concept reducing in-situ training load on memristor crossbar machine learning accelerators by up to 90\% using \chref{https://github.com/coreylammie/MemTorch}{MemTorch} and PyTorch.
% expand on this and have two copies depending on job
%   \item Researched novel input encoding and gradient thresholding methods for optimizing memristor crossbar machine learning
% accelerator in-situ performance using MemTorch and PyTorch
% \item First author paper on novel ‘thresholding’ concept which reduces the demand on crossbar devices by up to 90%, greatly
% improving longevity and reducing power consumption; currently pending submission
\end{itemize}

% \vspace{-0.2cm}

\cvsection{Skills}{}
\begin{itemize}
  \item \textbf{Languages}:
    c++, python, c, go, rust, 
    lua, javascript, html5, css, java, bash,
    SQL, verilog, MATLAB/simulink, assembly
   \item \textbf{Frameworks \& Libraries}:
     ROS, ROS2, numpy, scipy, OpenCV, Pandas,
     Jenkins, CI/CD, Docker, LXD, % LXC/LXD,
     % redis,  
     flask, Django, Android,
     % Android, % mobile,
     PyTorch, Tensorflow, Keras, TensorRT, CUDA,
     PostgreSQL, MySQL, MongoDB,
     NodeJS, VueJS, ThreeJS,
     FPGA,  Cloud,
     AWS, GCP, git
   \item \textbf{Other}: Linux, UNIX, vim, debugging,
     object-oriented programming, embedded,
     systems software, 
     infrastructure, databases,
     REST APIs, MapReduce,
     user experience,
     % arduino,
     Fusion360,
     Googling, 
     % learning on-the-fly,
     technical writing and communication

     % TODO(ihasdapie): Demonstrate able to craft multi-functional requirements and translate them into practical engineering t
\end{itemize}

% \cvsection{Programming Languages}
% \cvtag{Python} \cvtag{c/c++}  \cvtag{rust}
% \cvtag{SQL} \cvtag{bash} \cvtag{MATLAB} \cvtag{Simulink}
% \cvtag{assembly} \cvtag{verilog}
% \cvtag{CSS} \cvtag{HTML}

% \cvsection{Tools \& Libraries}
% \cvtag{Googling} \cvtag{Django} \cvtag{PyTorch} \cvtag{Keras} \cvtag{Cloud computing}
% \cvtag{Android} \cvtag{Git} \cvtag{Docker} 
% \cvtag{PostgreSQL} \cvtag{MongoDB} \cvtag{node} \cvtag{Vue.js}
% \cvtag{ROS/ROS2} \cvtag{UNIX tools}  
% \cvtag{\LaTeX} \cvtag{CAD} \cvtag{FPGA}

% \cvsection{Interests}
% \cvtag{FOSS} \cvtag{Linux} \cvtag{\texttt{vim}}
% \cvtag{Reliability} \cvtag{AI} \cvtag{Autonomous Vehicles} 
% \cvtag{Badminton}



\cvsection{Projects, Awards, \& More}{\textit{For demos, please see \chref{https://chenbrian.ca/posts/projects/}{chenbrian.ca/posts/projects}}}

\begin{itemize}
  \item \textbf{\chref{https://github.com/btrnt}{\textit{``butternut"}}}: Implementing \chref{http://gltr.io/}{gltr} on \chref{https://github.com/salesforce/ctrl}{CTRL} to combat AI-generated text. nwHacks bronze, KPMG Data Analysis \& Salesforce Award.
  \item \textbf{\chref{https://github.com/UTMIST/humerus}{\textit{``the Humerus Bot"}}}: Directed project with \chref{https://utmist.gitlab.io/}{UTMIST} to build a NLP bot designed to \textit{win} Cards Against Humanity
  \item \textbf{\chref{chenbrian.ca/posts/2021/teaching/}{Teaching}}: Review content I prepared for my students, including a \chref{https://mybinder.org/v2/gh/ihasdapie/teaching/HEAD}{custom Jupyter notebook} with c kernel for interactive learning
  \item \textbf{Awards}: Schulich Leadership Scholarship nominee, Bert \& Greta Quartermaine Badminton Scholarship Recipient, BC District Scholarship \& BC Achievement Scholarship Recipient, Canada Service Corps Student Service Grant, ESROP-UofT research grant
  % \item \textbf{\chref{https://github.com/ihasdapie/dotfiles}{\textit{dotfiles}}}: my extensive Linux user application and \chref{https://github.com/ihasdapie/dotfiles/tree/main/nvim}{neovim} configurations with various in-process plugins
  \item \textbf{Badminton}: ClearOne Nationals Team, 2018 Junior Nationals Finalist, Eric Hamber Provincial Team Captain, \chref{https://www.uoftbadmintonclub.com/}{UTBC} Exec
  \item \textbf{Theatre}: Wrote and directed full-length show: \textit{`To Bleach a Pigeon'}. Oversaw actors, crew, set design, and creative process
% \item \textbf{\textit{``SigVer"}}, a >90\% accurate \textbf{Siamese} neural network for secure signature verification
\end{itemize}









% \cvsection{Referees}

% \cvref{name}{email}{mailing address}
% \cvref{Prof.\ Alpha Beta}{Institute}{a.beta@university.edu}
% {Address Line 1\\Address line 2}


\end{document}
