\documentclass[10pt,a4paper,ragged2e,withhyper]{altacv}
%% and packages.
%% See http://texdoc.net/pkg/fontawesome5 and http://texdoc.net/pkg/academicons for full list of symbols. You MUST compile with XeLaTeX or LuaLaTeX if you want to use academicons.

% Change the page layout if you need to
\geometry{left=1.5cm,right=1.5cm,top=1.1cm,bottom=1cm}

% The paracol package lets you typeset columns of text in parallel
\usepackage{paracol}


% Change the font if you want to, depending on whether
% you're using pdflatex or xelatex/lualatex
\ifxetexorluatex
  % If using xelatex or lualatex:
  \setmainfont{Lato}
\else
  % If using pdflatex:
  \usepackage[default]{lato}
\fi

% Change the colours if you want to
\definecolor{MintyGreen}{HTML}{33b884} %{3E0097}
\definecolor{SlateGrey}{HTML}{2E2E2E}
\definecolor{LightGrey}{HTML}{666666}
% \colorlet{name}{black}
\colorlet{tagline}{MintyGreen}
\colorlet{heading}{MintyGreen} \colorlet{headingrule}{MintyGreen}
% \colorlet{subheading}{PastelRed}
\colorlet{accent}{MintyGreen}
\colorlet{emphasis}{SlateGrey}
\colorlet{body}{LightGrey}


% make pdftotxt output nicer
\input{glyphtounicode}
\pdfgentounicode=1

\newcommand{\chref}[3][teal]{\href{#2}{\color{#1}{#3}}}%


% Change some fonts, if necessary
% \renewcommand{\namefont}{\Huge\rmfamily\bfseries}
% \renewcommand{\personalinfofont}{\footnotesize}
% \renewcommand{\cvsectionfont}{\LARGE\rmfamily\bfseries}
% \renewcommand{\cvsubsectionfont}{\large\bfseries}

% Change the bullets for itemize and rating marker
% for \cvskill if you want to
\renewcommand{\itemmarker}{{\small\textbullet}}
\renewcommand{\ratingmarker}{\faCircle}

%% sample.bib contains your publications

\hbadness = 10001
\vbadness = 10001
\hfuzz=\maxdimen \tolerance=10000

\begin{document}
\name{Brian Chen}
\tagline{Student \& Aspiring Software Engineer}
%% You can add multiple photos on the left or right
% \photoL{2cm}{Yacht_High,Suitcase_High}
\personalinfo{%
  \homepage{chenbrian.ca}
  \linkedin{linkedin.com/in/brianchen28914}
  \github{github.com/ihasdapie} 
  \address{Vancouver, BC \& Toronto, ON Canada}
  \email{brianchen.chen [at] mail.utoronto.ca}
  \phone{+x xxx xxx xxxx}
  %% \printinfo{symbol}{detail}[optional hyperlink prefix]
}

\makecvheader


%% Depending on your tastes, you may want to make fonts of itemize environments slightly smaller
\AtBeginEnvironment{itemize}{\small}

\raggedright


\cvsection{Summary}
Current Engineering Science/Electrical \& Computer Engineering student at the University of Toronto with a demonstrated passion for software engineering.
A big Linux \& command line aficionado and avid badminton player.
I have worked in development and leadership positions in industry, research, and my own non-profit startup. 
Interested in working in a community to identify and resolve problems in the world around us.


\cvsection{Education}
\cvevent{B.A.Sc in Engineering Science}{University of Toronto}{2020 -- 2025}{Toronto, ON}
Electrical and Computer Engineering major, Machine Intelligence minor. cGPA 3.73, Dean's List.

% \cvevent{}{Langara College}{}{}
% Concurrent enrolment with studies at Eric Hamber Secondary
% \cvevent{}{Eric Hamber Secondary}{2015 -- 2020}{Vancouver, BC}

\cvsection{Experience}

\cvevent{Software Developer}{BC Parks Foundation}{July 2020 -- September 2021}{Vancouver, BC}
\begin{itemize}
  \item Coordinated with stakeholders to design and implement novel `DiscoverParks' platform and data collection/visualization solution for parks in British Columbia with \itag{python}, \itag{Django}, \itag{PostgreSQL}, \itag{VueJS}, \itag{docker}, and \itag{AWS}; currently in private beta.
  \item Built and maintained site backend, internal content management interface, and early-stage front-end experiences
  \item Identified and addressed two key bottlenecks in content management strategy, \textbf{improved efficiency by >10x}.

\end{itemize}

\cvevent{Teaching Assistant}{Division of Engineering Science - University of Toronto}{September 2021 - Present} {Toronto, ON}
\begin{itemize}
  \item Helped teach ESC180: Introduction to Programming and ESC190: Algorithms and Data Structures
  \item Prepared tutorial content and lead tutorials of 25 students; assisted labs and course evaluations. Review content is public at \chref{https://chenbrian.ca/posts/2021/teaching/}{chenbrian.ca/posts/2021/teaching}

\end{itemize}

\cvevent{Research Intern}{Intelligent Sensory Microsystems Lab - University of Toronto} {May 2021 -- Present} {Toronto, ON}
\begin{itemize}
  \item Researched novel input encoding and gradient thresholding methods for optimizing \itag{memristor} crossbar machine learning accelerator in-situ performance using \itag{\chref{https://github.com/coreylammie/MemTorch}{MemTorch}} and \itag{PyTorch}.
  \item Identified novel `thresholding' concept which \textbf{reduces the demand on crossbar devices by up to 90\%}, greatly improving longevity and reducing power consumption
  \item \itag{First author paper} on the aforementioned research pending submission
\end{itemize}

\cvevent{GrocerCheck Website; Co-Founder \& Developer}{GrocerCheck Foundation}{April 2020 -- December 2020}{}
\begin{itemize}
  \item Created \chref{https://grocercheck.ca}{grocercheck.ca}, a website that aggregates and visualizes grocery story busyness to help users shop more safely for groceries across \textbf{>15,000} stores in 10+ major cities
  \item Founded GrocerCheck Foundation, a \textbf{registered non-profit} to better scale project; secured support, funding, grants, and partnerships valued at \textbf{>\$200,000}, supporting >20,000 daily users.
  \item Designed and implemented custom \chref{https://github.com/GrocerCheck/LivePopularTimes}{LivePopularTimes} scraping library to power backend
\end{itemize}






\cvevent{Simulation \& Testing Co-lead}{aUToronto - UofT's Self Driving Car Team}{September 2020 -- Present}{Toronto, ON}
\begin{itemize}
  \item Leading multidisciplinary team of 14 students across 4 project groups to develop superior automated tooling for autonomous vehicle development. Our team, \chref{https://www.autodrive.utoronto.ca/}{aUToronto} has won the SAE Autodrive Challenge for four consecutive years.
  \item \textbf{\textit{"aUToTest"}}, automated simulation integration test framework for autonomous vehicles, with \itag{python}, \itag{matlab}, \itag{simulink}, \itag{docker}, \itag{ROS/ROS2}, and \itag{unreal engine}, enabling asynchronous testing \& reducing developer testing time by >1000\%
  \item \textbf{\textit{"aUToNoise"}} Machine learning - augmented sensor noise modelling for improved Sim2Real transfer using \itag{CycleGAN}
  \item \textbf{\textit{"aUToViz"}} test result visualization framework and \itag{Jenkins}/\itag{GitLab} \itag{CI/CD} integrations for aUToronto software stack
  \item Presented work at 2021 Vector Institute Mobility Symposium \& 2021 UofT Robotics Institute AV workshop

\end{itemize}

\cvsection{Projects/Other}

\begin{itemize}
  \item For more project information and demos please visit \chref{https://chenbrian.ca/posts/2021/projects/}{chenbrian.ca/posts/2021/projects}
  \item \textbf{\chref{https://github.com/btrnt}{\textit{``butternut"}}}, a chrome extension implementing \chref{http://gltr.io/}{gltr} that detects AI-generated text. nwHacks bronze, KPMG Data Analysis \& Groundswell Salesforce Award
  \item \textbf{\chref{https://github.com/UTMIST/humerus}{\textit{``the Humerus Bot"}}}, an applied \itag{NLP}{Natural Language Processing} project to write a bot that can \textit{win} Cards Against Humanity
  \item \textbf{\chref{https://github.com/ihasdapie/dotfiles}{\textit{dotfiles}}}, my extensive Linux user application and \chref{https://github.com/ihasdapie/dotfiles/tree/main/nvim}{neovim} configurations with various in-process plugins
  \item \textbf{Badminton}: ClearOne Nationals Team, Eric Hamber Provincial Team Captain \& Assistant Coach, 2018 Junior Nationals Finalist, UofT Badminton Club Exec
  \item \textbf{Theatre}: Wrote and directed full-length show: \textit{`To Bleach a Pigeon'}. Oversaw actors, crew, set design, and creative process
  \item \textbf{Awards}: Schulich Leadership Scholarship nominee, Bert \& Greta Quartermaine Badminton Scholarship Recipient, BC District Scholarship \& BC Achievement Scholarship Recipient, Canada Service Corps Student Service Grant, ESROP-UofT
% \item \textbf{\textit{``SigVer"}}, a >90\% accurate \textbf{Siamese} neural network for secure signature verification
\end{itemize}




% \cvsection{Programming Languages}
% \cvtag{Python} \cvtag{c/c++}  \cvtag{rust}
% \cvtag{SQL} \cvtag{bash} \cvtag{MATLAB} \cvtag{Simulink}
% \cvtag{assembly} \cvtag{verilog}
% \cvtag{CSS} \cvtag{HTML}

% \cvsection{Tools \& Libraries}
% \cvtag{Googling} \cvtag{Django} \cvtag{PyTorch} \cvtag{Keras} \cvtag{Cloud computing}
% \cvtag{Android} \cvtag{Git} \cvtag{Docker} 
% \cvtag{PostgreSQL} \cvtag{MongoDB} \cvtag{node} \cvtag{Vue.js}
% \cvtag{ROS/ROS2} \cvtag{UNIX tools}  
% \cvtag{\LaTeX} \cvtag{CAD} \cvtag{FPGA}

% \cvsection{Interests}
% \cvtag{FOSS} \cvtag{Linux} \cvtag{\texttt{vim}}
% \cvtag{Reliability} \cvtag{AI} \cvtag{Autonomous Vehicles} 
% \cvtag{Badminton}



% \cvsection{Referees}

% \cvref{name}{email}{mailing address}
% \cvref{Prof.\ Alpha Beta}{Institute}{a.beta@university.edu}
% {Address Line 1\\Address line 2}


\end{document}
